\documentclass[a4paper,12pt]{article}
%\documentstyle[10pt]{article}
\usepackage[spanish]{babel} %para que ponga la fecha en castellano
\usepackage{fancyhdr}
\usepackage{tikz}
\usepackage{setspace}
\usepackage{graphics}
\usepackage{graphicx}
\usepackage{colortbl}
%\usepackage[draft]{hyperref}
\usepackage{hyperref}
\oddsidemargin -0.3 true cm
\topmargin -1.0 true cm
\textheight 23.5 true cm
\textwidth 16.5 true cm
%\onehalfspacing
\begin{document}
\spanishsignitems
\definecolor{grey}{rgb}{0.9,0.9,0.9} %gris


%gestyle{fancy}
\pagestyle{fancy}
\fancyfoot[C]{} % para que en el pie no ponga los numeros de pagina
\headheight=85pt %para cambiar el tamaño del encabezado
\fancyhead[L] %la "L" indica a la izquierda
{
\begin{minipage}{2.0cm}
\includegraphics[width=0.9\textwidth]{logouclm_1_bn.jpg}
%\end{minipage}
%\begin{minipage}{5.5cm}
%\tiny %tamaño de letra pequeño
%{
%\textsf
%{
%Universidad de Castilla-La Mancha\\
%Escuela de Arquitectura de Toledo}}
\end{minipage}}
\fancyhead[C] %la "C" indica al centro
{
\textsf{\textbf{RADIACI\'ON Y RUIDO\\CURSO 2018-2019}\\
\vspace{0.2cm}
%Area de F\'{\i}sica de la Tierra\\
%Pr\'actica 2\\
\vspace{0.2cm}
\underline{C\'alculo de diversos \'indices de ruido en una zona urbana}
\underline{de Toledo a lo largo de una semana tipo}
}} %textsf es un tipo de letra

\fancyhead[R] %la "R" indica a la derecha
{
\small
{
\textsf
{
Profesores\\ \textbf{Enrique S\'anchez}\\
\textbf{Clemente Gallardo}\\
\textbf{Claudia Guti\'errez}
}}}

\vspace{1.5cm}

\section*{Introducci\'on te\'orica:}
Experimentalmente se comprueba que las variaciones de sensaci\'on sonora no son 
proporcionales a las variaciones de intensidad del sonido que llega al o\'{\i}do. 
Es decir, si al o\'{\i}do llegan dos est\'{\i}mulos tales que uno es el doble que el otro, 
las dos sensaciones que inducen no guardan la misma proporci\'on entre ellas. De hecho, 
el est\'{\i}mulo ($E$) y la sensaci\'on ($S$) est\'an relacionados por la llamada ley de 
Weber-Fechner, que se expresa por la ecuaci\'on:

\begin{equation}
S=k \cdot log E
\end{equation}

Derivando esta ecuaci\'on se comprueba que las variaciones de sensaciones son 
proporcionales a los incrementos relativos ($dE$) de los est\'{\i}mulos:

\begin{equation}
dS=k \frac{dE}{E}
\end{equation}

Por eso, en ac\'ustica se utiliza una escala de intensidades relativas, utilizando como 
unidad el Belio, que se define como el incremento que corresponde a una relaci\'on de 
intensidades de 1:10, es decir:

\begin{equation}
1 Belio = log\frac{10 I_i}{I_i}
\end{equation}

Lo que implica que el n\'umero de Belios, correspondiente a una relaci\'on de 
intensidades $I_2/I_1$, se determina mediante la expresi\'on:

\begin{equation}
n^o de Belios = log\frac{I_2}{I_1}
\end{equation}

En consecuencia, se define una escala logar\'{\i}tmica que relaciona intensidades relativas. 
Entonces, eligiendo una intensidad de referencia $I_0$ = 10$^{-12}$ W$\cdot$m$^{-2}$ 
(que aproximadamente corresponde al umbral de percepci\'on de un sonido de frecuencia 
1000 Hz) se obtiene una escala absoluta de intensidades dada por:

\begin{equation}
L_i = log\frac{I_i}{I_0}
\end{equation}

As\'{\i}, los valores extremos son:

M\'aximo ($M$):
\begin{equation}
L_M = log\frac{I_M}{I_0}=log \frac{100 \; W \cdot m^{-2}}{10^{-12}\;  W \cdot m^{-2}}=14 \; Belios 
\end{equation}

M\'{\i}nimo ($m$):
\begin{equation}
L_m = log\frac{I_0}{I_0}=log \; 1=0 \; Belios 
\end{equation}

De esta manera se obtiene una escala de niveles de intensidad ac\'ustica comprendido entre 
0 y 14 Belios, que es un intervalo menor que si se utilizara una escala en W$\cdot$m$^{-2}$
(de $10^{-12}$ a $10^2$). 
De todas formas, como el Belio es una unidad demasiado grande para las intensidades 
ac\'usticas normales, la unidad que se utiliza de forma general es el 
decibelio (dB) que es la d\'ecima parte del Belio. En consecuencia, el nivel absoluto de 
una intensidad $I_i$ expresado en dB ser\'a:

\begin{equation}
L_i = 10 \cdot log\frac{I_i}{I_0}
\end{equation}

La expresi\'on del nivel de intensidad puede adoptar distintas formas en 
funci\'on de los pa\-r\'a\-me\-tros que intervienen en su medida. As\'{\i}, se 
puede expresar en funci\'on de la potencia de la fuente ($W$), de la 
distancia ($r$) o de la presi\'on ($p$), de forma que:

\begin{equation}
L_i = 10 \cdot log\frac{I_i}{I_0}=10 \cdot log \frac{W_i / 4 \pi r^2}{W_0 / 4 \pi r^2}=
10 \cdot log \frac{W_i}{W_0}=10 \cdot log \frac{p_i^2/ \rho c}{p^2_0 / \rho c}=
20 \cdot \frac{p_i}{p_0}
\end{equation}

\subsection*{Operaciones con niveles ac\'usticos en decibelios}

Como la unidad dB tiene un car\'acter logar\'{\i}tmico, para operar con magnitudes que 
est\'an expresadas en dB hay que observar ciertas precauciones, relacionadas
con las operaciones logar\'{\i}tmicas.
La intensidad sonora $I_t$ resultado de varias fuentes $I_i$ se puede calcular
directamente mediante $I_t=\sum I_i$. Teniendo en cuenta que la definici\'on  de los logaritmos 
dice que $a=log\; b \Leftrightarrow 10^a=b$, podemos expresar la relaci\'on $L_i \Leftrightarrow I_i$
como

\begin{equation}
L_i = 10 \cdot log\frac{I_i}{I_0} \Leftrightarrow \frac{L_i}{10}= log\frac{I_i}{I_0}  
\Leftrightarrow 10^{\frac{L_i}{10}}=\frac{I_i}{I_0}
\end{equation}

Al pasar a la escala logar\'{\i}tmica, se obtiene lo siguiente:

%As\'{\i}, la suma 
%de niveles ac\'usticos en dB debidos a diversas fuentes de ruido se realiza de 
%la siguiente manera:


\begin{equation}
L_t= 10 \cdot log\frac{I_t}{I_0}=10 \cdot log \frac{\sum{I_i}}{I_0}=
10 \cdot log \sum(\frac{I_i}{I_0})= 10 \cdot log \sum 10^\frac{L_i}{10}
\end{equation}
%L_t=\sum L_i=\sum [  10 \cdot log\frac{I_i}{I_0}]=\sum[10 \cdot log (10^\frac{L_i}{10})]=
%10 \cdot log [\sum 10^\frac{L_i}{10} ]


Por ejemplo, el nivel de intensidad o presi\'on ac\'ustica de tres se\~nales de 
40, 50 y 60 dB resulta:

\begin{equation}
L_t=10 \cdot log(10^4+10^5+10^6)=60.45 dB
\end{equation}

Esto supone que el resultado de la suma de varias fuentes sonoras es un nivel 
ac\'ustico muy pr\'oximo al de la fuente m\'as intensa.


\section*{\'Indices ac\'usticos}

El objetivo de estos \'{\i}ndices es valorar de forma cuantitativa la posible 
molestia que provocan los ruidos ambientales de muy diversa procedencia, y por 
tanto de intensidades y frecuencias muy variables.

\subsection*{a) Nivel continuo equivalente ($L_{eq}$):}

Cuando se consideran niveles que var\'{\i}an temporalmente $L(t)$, como ser\'{\i}a 
por ejemplo el ruido de tr\'afico rodado, se utiliza el llamado nivel continuo 
equivalente ($L_{eq}$), que se define como el nivel en $dB$ correspondiente al promedio de 
la cantidad de energ\'{\i}a ac\'ustica variable durante un periodo de tiempo 
$T$, cuya ecuaci\'on matem\'atica es:

\begin{equation}
L_{eq}=10 \cdot log [ \frac{1}{T} \int_0^T 10^\frac{L(t)}{10} dt ]
\end{equation}

que se puede calcular de forma aproximada mediante la expresi\'on:

\begin{equation}
L_{eq}=10 \cdot log [ \frac{1}{T} \sum \left(t_i \cdot 10^\frac{L_i}{10} \right)]
\end{equation}

siendo $t_i$ los tiempos de observaci\'on durante los cuales el nivel sonoro $L_i$
se mantiene en un intervalo de $\pm$ 2 $dB$.

\underline{\it Ejemplo:}

?`Cu\'al es el nivel continuo equivalente correspondiente al ruido de tr\'afico observado 
a lo largo de un d\'{\i}a en una determinada calle urbana con la siguiente 
distribuci\'on de niveles sonoros: 85 dB durante 4 horas, 90 dB durante 2 horas, 
80 dB durante 10 horas y 65 dB durante 8 horas?

Soluci\'on:

\begin{equation}
L=10 \cdot log(\frac{4 \cdot 10^{8.5}+2 \cdot 10^9+ 10 \cdot 10^8 + 8 \cdot 10^{6.5}}{24})=82.52 dB
\end{equation}

\subsection*{b) Nivel d\'{\i}a-tarde-noche ($L_{den}$)}

Este criterio se establece a causa de la necesidad de evaluar de diferente forma 
los niveles de ruido urbano durante el d\'{\i}a, la tarde y la la noche, penalizando los 
ruidos nocturnos por estimar que la molestia que producen es mayor. Su 
valor se calcula con la ecuaci\'on:

\begin{equation}
	L_{den}=10 \cdot log[\frac{12 \cdot 10^{L_d/10}+ 4 \cdot 10^{(L_e+5)/10} + 8 \cdot 10^{(L_n+10)/10}}{24}]
\end{equation}

donde $L_d$ representa el nivel sonoro equivalente para el periodo de 
d\'{\i}a, $L_e$ para el periodo de tarde y  $L_n$ para el periodo nocturno.
Para saber exactamente los periodos correspondientes a cada tramo horario,
consulta la ordenanza municipal de Toledo, por ejemplo, en su art\'{\i}culo 55.3,
o en el Anexo I.A.1.

\subsection*{c) Niveles percentiles}

Considerando que el ruido urbano tiene un car\'acter aleatorio, resulta conveniente 
usar los \'{\i}ndices estad\'{\i}sticos llamados percentiles. Los que m\'as se utilizan son:
\begin{itemize}
\item $L_{10}:$ El nivel sonoro en dB que se sobrepasa durante el 10\% del tiempo de observaci\'on. 
Por lo general corresponde a los niveles m\'as elevados
\item $L_{90}:$ El nivel sonoro en dB que se sobrepasa durante el 90\% del tiempo de observaci\'on. Por lo general corresponde al nivel sonoro de fondo.
\end{itemize}
\subsection*{d) \'Indice de ruido de trafico ($TNI)$}
Es un \'{\i}ndice obtenido de la combinaci\'on ponderada de $L_{10}$ y $L_{90}$, promediadas 
a lo largo de un periodo de tiempo de 24 horas, de manera que:

\begin{equation}
TNI=4(L_{10}-L_{90})+L_{90}-30
\end{equation}

\section*{\underline{Objetivo de la pr\'actica}}

C\'alculo de los diversos \'{\i}ndices de ruido urbano correspondientes a una serie de medidas 
de niveles ac\'usticos registrados en la Plaza de Cuba de la ciudad de Toledo. Los 
valores que se muestran en las tablas proporcionadas corresponden a promedios 
diezminutales (en dB) a lo largo de una semana.

En un documento word responde a las siguientes cuestiones:
\begin{enumerate}
\item Calcula los siguientes \'{\i}ndices para cada d\'{\i}a de la semana : 
Nivel continuo equivalente ($L_{eq}$), nivel d\'{\i}a-tarde-noche ($L_{den}$),
		junto con los niveles de d\'{\i}a ($L_d$), de tarde ($L_e$) y de noche ($L_n$) y el \'{\i}ndice de 
ruido de tr\'afico ($TNI$) (adem\'as de los percentiles), y anota los resultados en una tabla como
la siguiente:

\begin{table}[h]
\begin{center}
\begin{tabular}{|c||@{\hspace{0.6cm}}c@{\hspace{0.6cm}}||@{\hspace{0.6cm}}c@{\hspace{0.6cm}}|@{\hspace{0.6cm}}c@{\hspace{0.6cm}}|@{\hspace{0.6cm}}c@{\hspace{0.6cm}}||@{\hspace{0.6cm}}c@{\hspace{0.6cm}}||@{\hspace{0.6cm}}c@{\hspace{0.6cm}}|@{\hspace{0.6cm}}c@{\hspace{0.6cm}}||@{\hspace{0.6cm}}c@{\hspace{0.6cm}}||}
\hline
	D\'{\i}a&\cellcolor{grey}{$L_{eq}$}&$L_d$&$L_e$&$L_n$&\cellcolor{grey}{$L_{den}$}&$L_{10}$&$L_{90}$&\cellcolor{grey}{$TNI$}\\
\hline
	Lunes&&&&&&&&\\
\hline
	Martes&&&&&&&&\\
\hline
	Mi\'ercoles&&&&&&&&\\
\hline
	Jueves&&&&&&&&\\
\hline
	Viernes&&&&&&&&\\
\hline
	S\'abado&&&&&&&&\\
\hline
	Domingo&&&&&&&&\\
\hline
\end{tabular}
\end{center}
\end{table}

\item Comenta estos resultados. Especula sobre las diferencias entre los diferentes valores
obtenidos para estos \'{\i}ndices, y la posible explicaci\'on sobre su evoluci\'on
a lo largo de la semana. Representa las series temporales directas por si te sirve de ayuda.
Intenta encontrar alg\'un otro \'{\i}ndice que pudiera expresar algo
parecido al TNI. Puedes consultar la normativa/ordenanza de ruido de la ciudad de Toledo
en \url{https://www.toledo.es/wp-content/uploads/2016/11/oca.pdf}, para completar tu
an\'alisis y ver si se superan los umbrales marcados o no, y comentarlo en tu documento.
\end{enumerate}
\end{document}
